\documentclass[a4paper, 12pt, english]{article}

% \usepackage[portuges]{babel}
\usepackage[utf8]{inputenc}
\usepackage{amsmath,amssymb}
\usepackage{graphicx}
\usepackage{subfig}
\usepackage[colorinlistoftodos]{todonotes}

\usepackage{indentfirst}
\usepackage{verbatim}
\usepackage{textcomp}
\usepackage{gensymb}
\usepackage{float}

\usepackage{relsize}

\usepackage{lipsum}% http://ctan.org/pkg/lipsum
\usepackage{xcolor}% http://ctan.org/pkg/xcolor
\usepackage{xparse}% http://ctan.org/pkg/xparse
\NewDocumentCommand{\myrule}{O{1pt} O{2pt} O{black}}{%
	\par\nobreak % don't break a page here
	\kern\the\prevdepth % don't take into account the depth of the preceding line
	\kern#2 % space before the rule
	{\color{#3}\hrule height #1 width\hsize} % the rule
	\kern#2 % space after the rule
	\nointerlineskip % no additional space after the rule
}
\usepackage[section]{placeins}

\usepackage{booktabs}
\usepackage{colortbl}%
\newcommand{\myrowcolour}{\rowcolor[gray]{0.925}}

\usepackage[obeyspaces]{url}
\usepackage{etoolbox}
\usepackage[colorlinks,citecolor=black,urlcolor=blue,bookmarks=false,hypertexnames=true]{hyperref}

\usepackage{geometry}
\geometry{
	paper=a4paper, % Change to letterpaper for US letter
	inner=3cm, % Inner margin
	outer=3cm, % Outer margin
	bindingoffset=.5cm, % Binding offset
	top=2cm, % Top margin
	bottom=2cm, % Bottom margin
	%showframe, % Uncomment to show how the type block is set on the page
}
\usepackage{fancyhdr}

% Define header and footer
\pagestyle{fancy}
\fancyhf{}
\lhead{ENER 104L}
\rhead{iSciM, Habib University} % Right-aligned page number in the header
\rfoot{\thepage} % Right footer text
%*******************************************************************************%
%************************************START**************************************%
%*******************************************************************************%
\begin{document}

%************************************TITLE PAGE**************************************%
\begin{titlepage}
	\begin{center}
		\textbf{\LARGE Habib University}\\[0.5cm]
		\textbf{\large iSciM}\\[0.2cm]
		\textbf {\large Fall 2023}\\[0.2cm]
		\vspace{20pt}
		\includegraphics[width=5cm]{../habiblogo.jpg}\\[1cm]
		\par
		\vspace{20pt}
		\textbf{\Large ENER 104L RENEWABLE ENERGY}\\
		\vspace{15pt}
		\myrule[1pt][7pt]
		\textbf{\LARGE  LABORATORY REPORT 7}\\
		\vspace{15pt}
		\textbf{\large Heat Energy from Solar Energy}\\
		\myrule[1pt][7pt]
		\vspace{25pt}
		\begin{tabular}{@{}p{5cm}p{3cm}@{}}
			\textbf{\large Student Name} & \textbf{\large Student ID} \\
			Ali Asghar Yousuf            & ay06993                    \\ % No1 
			Syed Ibrahim Ali Haider      & sh06565                    \\ % No2
		\end{tabular}

		\vspace{10pt}
		\begin{tabular}{@{}p{5cm}p{3cm}@{}}
			\textbf{\large Group Name} & \textbf{\large Group No.} \\
			Insane Fr                  & 1                         \\
		\end{tabular}

		\vspace{45pt}
		\textbf {\large Lab Instructors:}\\[0.2cm]
		\Large {Paishwa Naqvi}\\[0.1cm]
		\Large {Mah Noor Jamil}\\[0.1cm]
		\Large {Amber Talat}\\[0.1cm]
	\end{center}

	\par
	\vfill
	\begin{center}
		\textbf{\today}\\
	\end{center}

\end{titlepage}

%************************************TABLE OF CONTENTS**************************************%

%  %Sumário
%  \newpage
%  \tableofcontents
%  \thispagestyle{empty}
%  %End Sumário

%********************************%
%***********SECTION 1************%
%********************************%
\newpage
\section{Objectives}
\begin{itemize}
	\item To understand the concept of solar energy and its conversion to heat energy.
	\item To understand how the color of an object affects the amount of heat energy it
	      absorbs.
	\item To understand the role of insulation in reducing heat loss.
\end{itemize}

\section{Abstract}
Abstract

\section{Result and Analysis}
\subsection{Part A}
\begin{table}[H]
	\centering
	\caption{Temperature Table for Part A}
	\label{tab:table1}
	\begin{tabular}{@{}llll@{}}
		\toprule
		            & \textbf{$T_i$ (\degree C)} & \textbf{$T_f$ (\degree C)} & \textbf{$\Delta T$ (\degree C)} \\ \midrule
		White Plate & 21                         & 24                         & 3                               \\
		Black Plate & 22                         & 30                         & 8                               \\ \bottomrule
	\end{tabular}
\end{table}

The temperature of the white plate increased by 3\degree C while the
temperature of the black plate increased by 8\degree C. This is because the
white plate reflects most of the light falling on it while the black plate
absorbs most of the light falling on it. Therefore the temperature increase of
the black plate was higher than that of the white plate.
\subsection{Part B}
\begin{table}[H]
	\centering
	\caption{Temperature Table for Part B}
	\label{tab:table2}
	\begin{tabular}{@{}llll@{}}
		\toprule
		          & \textbf{$T_i$ (\degree C)} & \textbf{$T_f$ (\degree C)} & \textbf{$\Delta T$ (\degree C)} \\ \midrule
		Uncovered & 22                         & 30                         & 8                               \\
		Insulated & 23                         & 38                         & 15                              \\ \bottomrule
	\end{tabular}
\end{table}

The temperature of the uncovered plate increased by 8\degree C while the
temperature of the insulated plate increased by 15\degree C. This
is because the heat loss from the insulated plate was less than that of the
uncovered plate. Therefore the temperature increase of the insulated plate was
significantly higher than that of the uncovered plate.

\subsection{Part C}
$T_0$ represents the temperature of the plate without the plexiglass and $T_p$
represents the temperature of the plate with the plexiglass.
\begin{table}[H]
	\centering
	\caption{Temperature Table for Part C}
	\label{tab:table3}
	\begin{tabular}{@{}lll@{}}
		\toprule
		\textbf{Time (min)} & \textbf{$T_0$ (\degree C)} & \textbf{$T_p$ (\degree C)} \\ \midrule
		0                   & 22                         & 22                         \\
		1                   & 26                         & 27                         \\
		2                   & 30                         & 31                         \\
		3                   & 34                         & 35                         \\
		4                   & 38                         & 39                         \\
		5                   & 41                         & 42                         \\
		6                   & 43                         & 45                         \\
		7                   & 45                         & 47                         \\
		\bottomrule \multicolumn{3}{l}{Lamp turned off}                                          \\ \bottomrule
		8                   & 45                         & 47                         \\
		9                   & 43                         & 46                         \\
		10                  & 41                         & 44                         \\
		11                  & 39                         & 43                         \\
		12                  & 38                         & 42                         \\
		13                  & 37                         & 41                         \\
		14                  & 35                         & 40                         \\
		15                  & 34                         & 39                         \\
		\bottomrule
	\end{tabular}
\end{table}

The temperature of the plate without the plexiglass increased by 23\degree C
while the temperature of the plate with the plexiglass increased by 25\degree
C over 7 mins when the lamp was turned on. This is because the plexiglass simulated the greenhouse effect and trapped
the heat inside. Therefore the temperature increase of the plate with the
plexiglass was significantly higher than that of the plate without the
plexiglass.
\section{Conclusion}
abc

\end{document}