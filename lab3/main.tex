\documentclass[a4paper, 12pt, english]{article}

% \usepackage[portuges]{babel}
\usepackage[utf8]{inputenc}
\usepackage{amsmath,amssymb}
\usepackage{graphicx}
\usepackage{subfig}
\usepackage[colorinlistoftodos]{todonotes}

\usepackage{indentfirst}
\usepackage{verbatim}
\usepackage{textcomp}
\usepackage{gensymb}
\usepackage{float}

\usepackage{relsize}

\usepackage{lipsum}% http://ctan.org/pkg/lipsum
\usepackage{xcolor}% http://ctan.org/pkg/xcolor
\usepackage{xparse}% http://ctan.org/pkg/xparse
\NewDocumentCommand{\myrule}{O{1pt} O{2pt} O{black}}{%
	\par\nobreak % don't break a page here
	\kern\the\prevdepth % don't take into account the depth of the preceding line
	\kern#2 % space before the rule
	{\color{#3}\hrule height #1 width\hsize} % the rule
	\kern#2 % space after the rule
	\nointerlineskip % no additional space after the rule
}
\usepackage[section]{placeins}

\usepackage{booktabs}
\usepackage{colortbl}%
\newcommand{\myrowcolour}{\rowcolor[gray]{0.925}}

\usepackage[obeyspaces]{url}
\usepackage{etoolbox}
\usepackage[colorlinks,citecolor=black,urlcolor=blue,bookmarks=false,hypertexnames=true]{hyperref}

\usepackage{geometry}
\geometry{
	paper=a4paper, % Change to letterpaper for US letter
	inner=3cm, % Inner margin
	outer=3cm, % Outer margin
	bindingoffset=.5cm, % Binding offset
	top=2cm, % Top margin
	bottom=2cm, % Bottom margin
	%showframe, % Uncomment to show how the type block is set on the page
}
\usepackage{fancyhdr}

% Define header and footer
\pagestyle{fancy}
\fancyhf{}
\lhead{ENER 104L}
\rhead{iSciM, Habib University} % Right-aligned page number in the header
\rfoot{\thepage} % Right footer text
%*******************************************************************************%
%************************************START**************************************%
%*******************************************************************************%
\begin{document}

%************************************TITLE PAGE**************************************%
\begin{titlepage}
	\begin{center}
		\textbf{\LARGE Habib University}\\[0.5cm]
		\textbf{\large iSciM}\\[0.2cm]
		\textbf {\large Fall 2023}\\[0.2cm]
		\vspace{20pt}
		\includegraphics[width=5cm]{../habiblogo.jpg}\\[1cm]
		\par
		\vspace{20pt}
		\textbf{\Large ENER 104L RENEWABLE ENERGY}\\
		\vspace{15pt}
		\myrule[1pt][7pt]
		\textbf{\LARGE  LABORATORY REPORT 3}\\
		\vspace{15pt}
		\textbf{\large Calorimetry}\\
		\myrule[1pt][7pt]
		\vspace{25pt}
		\begin{tabular}{@{}p{5cm}p{3cm}@{}}
			\textbf{\large Student Name} & \textbf{\large Student ID} \\
			Ali Asghar Yousuf            & ay06993                    \\ % No1 
			Syed Ibrahim Ali Haider      & sh06565                    \\ % No2
		\end{tabular}

		\vspace{10pt}
		\begin{tabular}{@{}p{5cm}p{3cm}@{}}
			\textbf{\large Group Name} & \textbf{\large Group No.} \\
			Insane Fr                  & 1                         \\
		\end{tabular}

		\vspace{45pt}
		\textbf {\large Lab Instructors:}\\[0.2cm]
		\Large {Paishwa Naqvi}\\[0.1cm]
		\Large {Amber Talat}\\[0.1cm]
	\end{center}

	\par
	\vfill
	\begin{center}
		\textbf{\today}\\
	\end{center}

\end{titlepage}

%************************************TABLE OF CONTENTS**************************************%

%  %Sumário
%  \newpage
%  \tableofcontents
%  \thispagestyle{empty}
%  %End Sumário

%********************************%
%***********SECTION 1************%
%********************************%
\newpage
\section{Objectives}
\begin{itemize}
	\item To determine the potential energy content of different food items compare the
	      energy released by different food items.
	\item To conduct a calorimetry experiment to determine the energy content of food
	      items by capturing the energy released.
	\item To estimate the energy content of food items.
	\item To assess the heat content of food items.
	\item To compare the energy content of different food items.
\end{itemize}

\section{Abstract}
This report explores the energy content of different food items. It also
explores the energy release by different types of food items. The
energy content of different food items was determined by conducting a
calorimetry experiment. The calorimetry experiment was conducted by burning
different food items and capturing the energy released by the food items in
water. The temperature change of water was measured and the energy content of
the food items was calculated.

\section{Result and Analysis}
\subsection{Data}
\begin{table}[H]
	\caption{\label{tab:Table 1} Measurements}
	\centering
	\begin{tabular}{c c c c c c c}
		\toprule
		\textbf{Food}              &
		\textbf{$m_i$ (g)}         &
		\textbf{$m_f$ (g)}         &
		\textbf{$\Delta m$ (g)}    &
		\textbf{$T_i$ (\degree C)} &
		\textbf{$T_f$ (\degree C)} &
		\textbf{$\Delta T$ (\degree C)}                                \\

		\cmidrule[0.4pt](r{0.125em}){1-1}%
		\cmidrule[0.4pt](lr{0.125em}){2-2}%
		\cmidrule[0.4pt](lr{0.125em}){3-3}%
		\cmidrule[0.4pt](lr{0.125em}){4-4}%
		\cmidrule[0.4pt](lr{0.125em}){5-5}%
		\cmidrule[0.4pt](lr{0.125em}){6-6}%
		\cmidrule[0.4pt](l{0.25em}r{0.125em}){7-7}%
		% \midrule
		\textbf{Marshmallow}       & 1.25 & 0.80 & 0.45 & 22 & 32 & 10 \\
		\textbf{Chips}             & 0.50 & 0.09 & 0.41 & 22 & 42 & 20 \\
		\textbf{Cornflakes}        & 0.08 & 0.05 & 0.03 & 24 & 26 & 2  \\
		\textbf{White Beans}       & 0.23 & 0.22 & 0.01 & 24 & 25 & 1  \\
		\textbf{Red Beans}         & 0.37 & 0.34 & 0.03 & 23 & 23 & 0  \\
	\end{tabular}
\end{table}
We can observe that the carbohydrates and fats had the highest $\Delta M$
values.

\subsection{Calculations}
The formula for calculating the heat energy absorbed by water is given by:
\begin{equation*}
	H = m \times c \times \Delta T
\end{equation*}
where $m$ is the mass of water, $c$ is the specific heat capacity of water and
$\Delta T$ is the change in temperature of water.

The values of $m$ and $c$ are given as:
% left align equations
\begin{flalign*}
	\text{m} & = 100 \text{ g}            \\
	\text{c} & = 4.2 \text{ J/g\degree C}
\end{flalign*}
We can now calculate the energy content by different food items by using the
following formula:
\begin{equation*}
	E = \dfrac{H}{\Delta m}
\end{equation*}

\begin{table}[H]
	\caption{\label{tab:Table 2} Energy Content}
	\centering
	\begin{tabular}{c c c}
		\toprule
		\textbf{Food}                   &
		\textbf{$\Delta T$ (\degree C)} &
		\textbf{$E$ (kJ)}                            \\

		\cmidrule[0.4pt](r{0.125em}){1-1}%
		\cmidrule[0.4pt](lr{0.125em}){2-2}%
		\cmidrule[0.4pt](l{0.25em}r{0.125em}){3-3}%
		% \midrule
		\textbf{Marshmallow}            & 10 & 9.33  \\
		\textbf{Chips}                  & 20 & 20.49 \\
		\textbf{Cornflakes}             & 2  & 28.00 \\
		\textbf{White Beans}            & 1  & 42.00 \\
		\textbf{Red Beans}              & 0  & 0.00  \\
	\end{tabular}
\end{table}
\section{Conclusion}
We can conclude that the food items that are rich in carbohydrates are the ones
that release energy the quickest. Although carbohydrates release energy the
quickest, they do not release as much energy as fats. Although proteins are
also rich in energy, they do not release energy as quickly as carbohydrates.

\section{Questions}
\begin{enumerate}
	\item How could you improve your procedure to overcome difference in energy value?

	      \textbf{Answer:}
	      We could improve our procedure by using a more accurate thermometer and
	      measuring the temperature change more accurately. We could also increase the
	      number of trials to get a more accurate value.

	\item What is the name of the process in living organisms that releases energy from
	      food?

	      \textbf{Answer:}
	      The name of the process in living organisms that releases energy from food is
	      called cellular respiration.

	\item What are reactants and products in your experiment, write down in word form?

	      \textbf{Answer:}
	      The reactants in our experiment are the food items and the products are the
	      energy released by the food items.

	\item What are dependent and independent variables in this experiment?

	      \textbf{Answer:}
	      The dependent variable in this experiment is the energy released by the food
	      items and the independent variable is the food items.

	\item Which food items will release more energy? Are they most rich in fat,
	      carbohydrate, or protein? Which food burns well- protein-rich,
	      carbohydrate-rich, fat rich etc.

	      \textbf{Answer:}
	      The food items that will release more energy are the ones that are rich in
	      fats. The food items that are rich in carbohydrate will burn the quickest,
	      and the food items that are rich in fats will burn the slowest.

	\item Define what a calorie is and compare of low calorie and high calorie food items

	      \textbf{Answer:}
	      A calorie is a unit of energy, it is equal to 4.2 Joules. Low calorie food items have less energy content and high calorie food items have a higher energy content.

	\item Explain why body needs calories and what happens to excess calories?

	      \textbf{Answer:}
	      The body needs calories to perform its daily functions. Excess calories are stored in the body as fat.
\end{enumerate}

\end{document}