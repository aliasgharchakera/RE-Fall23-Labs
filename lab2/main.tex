\documentclass[a4paper, 12pt, english]{article}

% \usepackage[portuges]{babel}
\usepackage[utf8]{inputenc}
\usepackage{amsmath,amssymb}
\usepackage{graphicx}
\usepackage{subfig}
\usepackage[colorinlistoftodos]{todonotes}

\usepackage{indentfirst}
\usepackage{verbatim}
\usepackage{textcomp}
\usepackage{gensymb}
\usepackage{float}

\usepackage{relsize}

\usepackage{lipsum}% http://ctan.org/pkg/lipsum
\usepackage{xcolor}% http://ctan.org/pkg/xcolor
\usepackage{xparse}% http://ctan.org/pkg/xparse
\NewDocumentCommand{\myrule}{O{1pt} O{2pt} O{black}}{%
    \par\nobreak % don't break a page here
    \kern\the\prevdepth % don't take into account the depth of the preceding line
    \kern#2 % space before the rule
    {\color{#3}\hrule height #1 width\hsize} % the rule
    \kern#2 % space after the rule
    \nointerlineskip % no additional space after the rule
}
\usepackage[section]{placeins}

\usepackage{booktabs}
\usepackage{colortbl}%
\newcommand{\myrowcolour}{\rowcolor[gray]{0.925}}

\usepackage[obeyspaces]{url}
\usepackage{etoolbox}
\usepackage[colorlinks,citecolor=black,urlcolor=blue,bookmarks=false,hypertexnames=true]{hyperref}

\usepackage{geometry}
\geometry{
    paper=a4paper, % Change to letterpaper for US letter
    inner=3cm, % Inner margin
    outer=3cm, % Outer margin
    bindingoffset=.5cm, % Binding offset
    top=2cm, % Top margin
    bottom=2cm, % Bottom margin
    %showframe, % Uncomment to show how the type block is set on the page
}
\usepackage{fancyhdr}

% Define header and footer
\pagestyle{fancy}
\fancyhf{}
\lhead{ENER 104L}
\rhead{iSciM, Habib University} % Right-aligned page number in the header
\rfoot{\thepage} % Right footer text
%*******************************************************************************%
%************************************START**************************************%
%*******************************************************************************%
\begin{document}

%************************************TITLE PAGE**************************************%
\begin{titlepage}
    \begin{center}
        \textbf{\LARGE Habib University}\\[0.5cm]
        \textbf{\large iSciM}\\[0.2cm]
        \textbf {\large Fall 2023}\\[0.2cm]
        \vspace{20pt}
        \includegraphics[width=5cm]{../habiblogo.jpg}\\[1cm]
        \par
        \vspace{20pt}
        \textbf{\Large ENER 104L RENEWABLE ENERGY}\\
        \vspace{15pt}
        \myrule[1pt][7pt]
        \textbf{\LARGE  LABORATORY REPORT 2}\\
        \vspace{15pt}
        \textbf{\large Hydrogen Fuel Cell}\\
        \myrule[1pt][7pt]
        \vspace{25pt}
        \begin{tabular}{@{}p{5cm}p{3cm}@{}}
            \textbf{\large Student Name} & \textbf{\large Student ID} \\
            Ali Asghar Yousuf            & ay06993                    \\ % No1 
            Syed Ibrahim Ali Haider      & sh06565                    \\ % No2
        \end{tabular}

        \vspace{10pt}
        \begin{tabular}{@{}p{5cm}p{3cm}@{}}
            \textbf{\large Group Name} & \textbf{\large Group No.} \\
            Insane Fr                  & 1                         \\
        \end{tabular}

        \vspace{45pt}
        \textbf {\large Lab Instructors:}\\[0.2cm]
        \Large {Paishwa Naqvi}\\[0.1cm]
        \Large {Amber Talat}\\[0.1cm]
    \end{center}

    \par
    \vfill
    \begin{center}
        \textbf{\today}\\
    \end{center}

\end{titlepage}

%************************************TABLE OF CONTENTS**************************************%

%  %Sumário
%  \newpage
%  \tableofcontents
%  \thispagestyle{empty}
%  %End Sumário

%********************************%
%***********SECTION 1************%
%********************************%
\newpage
\section{Objectives}
\begin{itemize}
    \item Understand the working of a hydrogen fuel cell.
    \item Understand the process of electrolysis of water to produce hydrogen and oxygen.
    \item Generate electricity using a hydrogen and oxygen, understanding the working of
          a PEM fuel cell.
\end{itemize}

\section{Abstract}
Electricity generation using hydrogen fuel cells is a promising technology for
the future. It is a clean and efficient way of generating electricity. In this
lab, we will be using a PEM fuel cell to generate electricity. PEM fuel cells
are the most common type of fuel cells used today. PEM fuel cells use hydrogen
and oxygen to generate electricity. In this experiment, the hydrogen and oxygen
are produced by the electrolysis of water. The hydrogen and oxygen are then fed
into the fuel cell where they react to produce electricity and water as a
byproduct.

\section{Result and Analysis}
\subsection{Electrolysis of Water}
In this experiment, we used a PEM electrolyzer to electrolyze water. The PEM
electrolyzer uses electricity to split water into hydrogen and oxygen.

\begin{itemize}
    \item \textbf{Observation:} When the power supply was off, there were no bubbles in the containers.

          \textbf{Analysis:} No reaction taking place.
    \item \textbf{Observation:} After the power supply was turned on, tiny bubbles started forming in both the containers.

          \textbf{Analysis:} The water is being electrolyzed into hydrogen and oxygen.
    \item \textbf{Observation:} The bubbles became larger and the water started rising to the top containers.

          \textbf{Analysis:} The hydrogen and oxygen are being produced at a faster rate and pushing the water out of the lower containers.
    \item \textbf{Observation:} The bubble in one of the containers was larger than the other.

          \textbf{Analysis:} The container with the larger bubble was storing hydrogen and the other container was storing oxygen as the hydrogen and oxygen are produced in a 2:1 ratio in the electrolysis of water ($H_2O$).
\end{itemize}

\subsection{Electricity Generation using PEM Fuel Cell}
In this experiment, we used a PEM fuel cell to generate electricity. The PEM
fuel cell uses hydrogen and oxygen to generate electricity.

\begin{itemize}
    \item \textbf{Observation:} When the power supply was off, the fan was not spinning.

          \textbf{Analysis:} No electricity was being generated.
    \item \textbf{Observation:} When the power supply was turned on, the fan still wasn't spinning.

          \textbf{Analysis:} The fuel cell is not generating enough electricity to power the fan.
\end{itemize}

Even though the power supply was turned on, the fan was not spinning. This was
because the fuel cell was not generating enough electricity to power the fan.
So we connected the fuel cell to a multimeter to measure the voltage being
generated by the fuel cell.

\begin{itemize}
    \item \textbf{Observation:} When the power supply was turned on, the multimeter showed a voltage of 0.01V.

          \textbf{Analysis:} The fuel cell is generating electricity.

    \item \textbf{Observation:} Voltage was increased to 8V, the multimeter showed a voltage of 0.02V.

          \textbf{Analysis:} The fuel cell is generating more electricity, but not enough to power the fan.
\end{itemize}

Although the voltage was increased significantly from 2V to 8V, the voltage
measured by the multimeter only increased by 0.01V.

\section{Conclusion}
In this lab, we learned about the working of a hydrogen fuel cell. We learned
that hydrogen and oxygen are produced by the electrolysis of water, which was
successfully demonstrated in the lab. We also learned that hydrogen and oxygen
can be used to generate electricity using a PEM fuel cell, which despite not
being able to power the fan, was successfully demonstrated in the lab.

The PEM fuel cell is a promising technology for the future. It is a clean and
efficient way of generating electricity. It can be used to power cars, homes,
and even cities. It is a very versatile technology and can be used in many
different ways. 

\end{document}