\documentclass[a4paper, 12pt, english]{article}

% \usepackage[portuges]{babel}
\usepackage[utf8]{inputenc}
\usepackage{amsmath,amssymb}
\usepackage{graphicx}
\usepackage{subfig}
\usepackage[colorinlistoftodos]{todonotes}

\usepackage{indentfirst}
\usepackage{verbatim}
\usepackage{textcomp}
\usepackage{gensymb}

\usepackage{relsize}

\usepackage{lipsum}% http://ctan.org/pkg/lipsum
\usepackage{xcolor}% http://ctan.org/pkg/xcolor
\usepackage{xparse}% http://ctan.org/pkg/xparse
\NewDocumentCommand{\myrule}{O{1pt} O{2pt} O{black}}{%
	\par\nobreak % don't break a page here
	\kern\the\prevdepth % don't take into account the depth of the preceding line
	\kern#2 % space before the rule
	{\color{#3}\hrule height #1 width\hsize} % the rule
	\kern#2 % space after the rule
	\nointerlineskip % no additional space after the rule
}
\usepackage[section]{placeins}

\usepackage{booktabs}
\usepackage{colortbl}%
\newcommand{\myrowcolour}{\rowcolor[gray]{0.925}}

\usepackage[obeyspaces]{url}
\usepackage{etoolbox}
\usepackage[colorlinks,citecolor=black,urlcolor=blue,bookmarks=false,hypertexnames=true]{hyperref}

\usepackage{geometry}
\geometry{
	paper=a4paper, % Change to letterpaper for US letter
	inner=3cm, % Inner margin
	outer=3cm, % Outer margin
	bindingoffset=.5cm, % Binding offset
	top=2cm, % Top margin
	bottom=2cm, % Bottom margin
	%showframe, % Uncomment to show how the type block is set on the page
}
\usepackage{fancyhdr}

% Define header and footer
\pagestyle{fancy}
\fancyhf{}
\lhead{ENER 104L}
\rhead{iSciM, Habib University} % Right-aligned page number in the header
\rfoot{\thepage} % Right footer text
%*******************************************************************************%
%************************************START**************************************%
%*******************************************************************************%
\begin{document}

%************************************TITLE PAGE**************************************%
\begin{titlepage}
	\begin{center}
		\textbf{\LARGE Habib University}\\[0.5cm]
		\textbf{\large iSciM}\\[0.2cm]
		\textbf {\large Fall 2023}\\[0.2cm]
		\vspace{20pt}
		\includegraphics[width=5cm]{images/habiblogo.jpg}\\[1cm]
		\par
		\vspace{20pt}
		\textbf{\Large ENER 104L RENEWABLE ENERGY}\\
		\vspace{15pt}
		\myrule[1pt][7pt]
		\textbf{\LARGE  LABORATORY REPORT 1}\\
		\vspace{15pt}
		\textbf{\large Global Warming}\\
		\myrule[1pt][7pt]
		\vspace{25pt}
		\begin{tabular}{@{}p{5cm}p{3cm}@{}}
			\textbf{\large Student Name} & \textbf{\large Student ID} \\
			Ali Asghar Yousuf            & ay06993                    \\ % No1 
			Syed Ibrahim Ali Haider      & sh06565                    \\ % No2
		\end{tabular}

		\vspace{10pt}
		\begin{tabular}{@{}p{5cm}p{3cm}@{}}
			\textbf{\large Group Name} & \textbf{\large Group No.} \\
			Insane Fr                  & 1                         \\
		\end{tabular}

		\vspace{45pt}
		\textbf {\large Lab Instructors:}\\[0.2cm]
		\Large {Paishwa Naqvi}\\[0.1cm]
		\Large {Amber Talat}\\[0.1cm]
	\end{center}

	\par
	\vfill
	\begin{center}
		\textbf{\today}\\
	\end{center}

\end{titlepage}

%************************************TABLE OF CONTENTS**************************************%

%  %Sumário
%  \newpage
%  \tableofcontents
%  \thispagestyle{empty}
%  %End Sumário

%********************************%
%***********SECTION 1************%
%********************************%
\newpage
\section{Objectives}
\begin{itemize}
	\item Understand effect of various factors in our atmosphere.
	\item Understand that excess CO2 intensif ies the greenhouse effect
	\item Why is greenhouse effect important and what does it have to do with climate
	      change?
	\item Does greenhouse gases really make the temperature rise?
\end{itemize}
\section{Abstract}
This report explores the greenhouse effect's impact on Earth's atmosphere, taking into consideration of natural and human factors that fuel
global warming. A strong emphasis is created to reduce golbal warming pollution, this is done with the aid of practical experiments to us understand 
these complex processes. Part A focuses on dissecting the causes of global warming, with emphasis on the role of greenhouse gases. 
A hands-on experiment, based on a climate change by modeling our earth, is conducted to measure temperature fluctuations, so that we can foster a tangible understanding of this 
critical environmental issue. Part B delves into photosynthesis and respiration in plants, this experiments aids to quantify carbon dioxide and oxygen 
exchange. This helped us understand how life itself interacts with the environment, and enabled us to grasp a better concept
of Earth's ecosystems and the difficulties it faces. Overall, this report aims to provide a comprehensive understanding of the greenhouse effect and its impact on our planent. 
\section{Result and Analysis}
\subsection{Part I: The Greenhouse Effect}
\subsubsection{Temperature Graphs}
\begin{figure}[!ht]
	\centering
	\subfloat[Combined]{\includegraphics[width=0.4\columnwidth]{images/part1-graph.jpg}}\\
	\qquad
	\subfloat[Covered Jar]{\includegraphics[width=0.4\columnwidth]{images/part1-temp1-graph.jpg}}
	\qquad
	\subfloat[Uncovered Jar]{\includegraphics[width=0.4\columnwidth]{images/part1-temp2-graph.jpg}}
	\caption{Temperature Graphs}
	\label{fig:TempGraphs}
\end{figure}

\subsubsection{Temperature Table}

\begin{table}[!ht]
	\caption{\label{tab:Table 1} Temperature Table}
	\centering
	\begin{tabular}{c c c}
		\toprule
		  & \textbf{Covered Jar (\degree C)}
		  & \textbf{Uncovered Jar (\degree C)}        \\
		\cmidrule[0.4pt](r{0.125em}){1-1}%
		\cmidrule[0.4pt](lr{0.125em}){2-2}%
		\cmidrule[0.4pt](lr{0.125em}){3-3}%
		% \midrule
		\textbf{min}     & 28.4                    & 28.1                      \\
		\textbf{max}     & 29.7                    & 29.3                      \\
		\textbf{mean}    & 29.1                    & 28.6                      \\
		\textbf{st. dev} & 0.51668                 & 0.26998
	\end{tabular}
\end{table}

\subsection{Part II: Photosynthesis and Respiration}
\subsubsection{Covered Jar}
\paragraph{CO2 and O2 Graphs}

\begin{figure}[!ht]
	\centering
	\subfloat[CO2 and O2]{\includegraphics[width=0.4\columnwidth]{images/part2-covered-graph.jpg}}\\
	\qquad
	\subfloat[O2]{\includegraphics[width=0.4\columnwidth]{images/part2-covered-o2-graph.jpg}}
	\qquad
	\subfloat[CO2]{\includegraphics[width=0.4\columnwidth]{images/part2-covered-co2-graph.jpg}}
	\caption{Covered Jar}
	\label{fig:CoveredJar}
\end{figure}

\section{Conclusion}
abc

\section{Questions To Ponder}
\subsection{Part A:}

\begin{enumerate}
    \item Explain with reasons which beaker covered or uncovered has the greatest temperature change?
    \item Which beaker has the greatest rate of temperature change and why?
    \item What is slope and the rate of reaction?
    \item Why might the greenhouse effect be a problem for our earth?
    \item Did the model greenhouse warm faster or slower than the control? What do you think caused the difference?
    \item Describe one advantage of using a greenhouse.
\end{enumerate}

\subsection{Part B:}

\begin{enumerate}
    \item Were either of the rate values for CO2 a positive number? If so, what is the biological significance of this?
    \item Were either of the rate values for O2 a positive number? If so, what is the biological significance of this?
    \item Do you have evidence that photosynthesis occurred in leaves? Explain.
    \item Do you have evidence that respiration occurred in leaves? Explain.
\end{enumerate}


\begin{table}[]
	\begin{tabular}{lll}
		        & Covered Jar (\degree C) & Uncovered Jar (\degree C) \\
		min     & 28.4                    & 28.1                      \\
		max     & 29.7                    & 29.3                      \\
		mean    & 29.1                    & 28.6                      \\
		st. dev & 0.51668                 & 0.26998
	\end{tabular}
\end{table}
\begin{table}
	\caption[short]{\label{tab:Table 1} Experimental result for First order low pass filter}
	\centering
	\begin{tabular}{c c c c c}
		\toprule
		\multicolumn{4}{l|}{Chybovost \%} & \textbf{temp2}                \\
		\cmidrule[0.4pt](r{0.125em}){1-1}%
		\cmidrule[0.4pt](lr{0.125em}){2-2}%
		\cmidrule[0.4pt](lr{0.125em}){3-3}%
		\cmidrule[0.4pt](lr{0.125em}){4-4}%
		\cmidrule[0.4pt](lr{0.125em}){5-5}%
		10                                & 10             & 10 & 10 & 10

	\end{tabular}
\end{table}

%****************************************Table 1******************************************%
\begin{table}[!ht]
	\caption{\label{tab:Table 1} Experimental result for Second order low pass filter}
	\centering
	\begin{tabular}{c c c c c}
		\toprule
		\textbf{$ V_{in} (mV)$}
		     & \textbf{$ f $}
		     & \textbf{$V_{2} (mV)$}
		     & \textbf{(I.L)dB}
		     & \textbf{Phase ($\phi $)}                                   \\
		\cmidrule[0.4pt](r{0.125em}){1-1}%
		\cmidrule[0.4pt](lr{0.125em}){2-2}%
		\cmidrule[0.4pt](lr{0.125em}){3-3}%
		\cmidrule[0.4pt](lr{0.125em}){4-4}%
		\cmidrule[0.4pt](lr{0.125em}){5-5}%
		% \midrule
		5760 & 100 Hz                   & 1190   & -13.70 & $-5^{\circ}$  \\
		\myrowcolour%
		5730 & 200 Hz                   & 1160   & -13.87 & $-11^{\circ}$ \\
		5630 & 500 Hz                   & 1070   & -14.42 & $-21^{\circ}$ \\
		\myrowcolour%
		5320 & 1 kHz                    & 849    & -15.94 & $-38^{\circ}$ \\
		4930 & 2 kHz                    & 560    & -18.89 & $-65^{\circ}$ \\
		\myrowcolour%
		4830 & 5 kHz                    & 265.1  & -25.21 & $-74^{\circ}$ \\
		4780 & 10 kHz                   & 123.86 & -31.73 & $-77^{\circ}$ \\
		\myrowcolour%
		4700 & 20 kHz                   & 32.55  & -43.19 & $-79^{\circ}$ \\
		4700 & 50 kHz                   & 7.7    & -55.7  & $-83^{\circ}$ \\
		\myrowcolour%
		4700 & 100 kHz                  & 2.29   & -63.26 & $-85^{\circ}$ \\
		\bottomrule                                                       \\
	\end{tabular}
\end{table}
\end{document}