\documentclass[a4paper, 12pt, english]{article}

% \usepackage[portuges]{babel}
\usepackage[utf8]{inputenc}
\usepackage{amsmath,amssymb}
\usepackage{graphicx}
\usepackage{subfig}
\usepackage[colorinlistoftodos]{todonotes}

\usepackage{indentfirst}
\usepackage{verbatim}
\usepackage{textcomp}
\usepackage{gensymb}
\usepackage{float}

\usepackage{relsize}

\usepackage{lipsum}% http://ctan.org/pkg/lipsum
\usepackage{xcolor}% http://ctan.org/pkg/xcolor
\usepackage{xparse}% http://ctan.org/pkg/xparse
\NewDocumentCommand{\myrule}{O{1pt} O{2pt} O{black}}{%
	\par\nobreak % don't break a page here
	\kern\the\prevdepth % don't take into account the depth of the preceding line
	\kern#2 % space before the rule
	{\color{#3}\hrule height #1 width\hsize} % the rule
	\kern#2 % space after the rule
	\nointerlineskip % no additional space after the rule
}
\usepackage[section]{placeins}

\usepackage{booktabs}
\usepackage{colortbl}%
\newcommand{\myrowcolour}{\rowcolor[gray]{0.925}}

\usepackage[obeyspaces]{url}
\usepackage{etoolbox}
\usepackage[colorlinks,citecolor=black,urlcolor=blue,bookmarks=false,hypertexnames=true]{hyperref}

\usepackage{geometry}
\geometry{
	paper=a4paper, % Change to letterpaper for US letter
	inner=3cm, % Inner margin
	outer=3cm, % Outer margin
	bindingoffset=.5cm, % Binding offset
	top=2cm, % Top margin
	bottom=2cm, % Bottom margin
	%showframe, % Uncomment to show how the type block is set on the page
}
\usepackage{fancyhdr}

% Define header and footer
\pagestyle{fancy}
\fancyhf{}
\lhead{ENER 104L}
\rhead{iSciM, Habib University} % Right-aligned page number in the header
\rfoot{\thepage} % Right footer text
%*******************************************************************************%
%************************************START**************************************%
%*******************************************************************************%
\begin{document}

%************************************TITLE PAGE**************************************%
\begin{titlepage}
	\begin{center}
		\textbf{\LARGE Habib University}\\[0.5cm]
		\textbf{\large iSciM}\\[0.2cm]
		\textbf {\large Fall 2023}\\[0.2cm]
		\vspace{20pt}
		\includegraphics[width=5cm]{../habiblogo.jpg}\\[1cm]
		\par
		\vspace{20pt}
		\textbf{\Large ENER 104L RENEWABLE ENERGY}\\
		\vspace{15pt}
		\myrule[1pt][7pt]
		\textbf{\LARGE  LABORATORY REPORT 11}\\
		\vspace{15pt}
		\textbf{\large DSSC}\\
		\myrule[1pt][7pt]
		\vspace{25pt}
		\begin{tabular}{@{}p{5cm}p{3cm}@{}}
			\textbf{\large Student Name} & \textbf{\large Student ID} \\
			Ali Asghar Yousuf            & ay06993                    \\ % No1 
			Syed Ibrahim Ali Haider      & sh06565                    \\ % No2
		\end{tabular}

		\vspace{10pt}
		\begin{tabular}{@{}p{5cm}p{3cm}@{}}
			\textbf{\large Group Name} & \textbf{\large Group No.} \\
			Insane Fr                  & 1                         \\
		\end{tabular}

		\vspace{45pt}
		\textbf {\large Lab Instructors:}\\[0.2cm]
		\Large {Paishwa Naqvi}\\[0.1cm]
		\Large {Mah Noor Jamil}\\[0.1cm]
		\Large {Amber Talat}\\[0.1cm]
	\end{center}

	\par
	\vfill
	\begin{center}
		\textbf{\today}\\
	\end{center}

\end{titlepage}

%************************************TABLE OF CONTENTS**************************************%

%  %Sumário
%  \newpage
%  \tableofcontents
%  \thispagestyle{empty}
%  %End Sumário

%********************************%
%***********SECTION 1************%
%********************************%
\newpage
\section{Objectives}
\begin{itemize}
	\item Create a Dye Sensitized Solar Cell (DSSC)
	\item Understand the working of a DSSC
	\item Test the DSSC under different light sources
\end{itemize}

\section{Abstract}
Solar cells are devices that convert light energy into electrical energy. They
are made up of semiconductors that absorb light and convert it into electrical
energy. Dye Sensitized Solar Cells (DSSC) are a type of solar cell that use dye
to absorb light. In this experiment, we built a DSSC using a glass slide coated
with Indium Tin Oxide (ITO) on one side. The ITO side was coated with titanium
dioxide (TiO$_2$) and the other slide was coated with carbon. The slides were
then sandwiched together with a layer of organic dye in between.

\section{Result and Analysis}
The resistance of the conductive side of the ITO slide was $31.3 \text{ k}
	\Omega$ and the organic dye used was leaves juice. The DSSC was tested under
different light sources to measure the voltage, current and power produced. The
results are shown in Table \ref{tab:results}.

\begin{table}[H]
	\centering
	\caption{Table of results}
	\label{tab:results}
	\begin{tabular}{llll}
		\toprule
		\textbf{Light Source} & \textbf{Voltage (V)} & \textbf{Current (mA)} & \textbf{Power (mW)} \\
		\midrule
		\textbf{Ambient}      & 0.17                 & 0.1                   & 0.017               \\
		\textbf{Sunlight}     & 0.20                 & 0.1                   & 0.020               \\
		\textbf{Tungsten}     & 0.40                 & 0.1                   & 0.040               \\
		\bottomrule
	\end{tabular}
\end{table}

The current produced by the DSSC was the same for all light sources at 0.1 mA
which is a very small amount of current. But the voltage and power produced
increased with the intensity of the light source. The DSSC produced the highest
voltage under tungsten light at 0.40V which is a peculiar result, as sunlight
is a much more intense light source than tungsten. This could be due to the
time of the year as the experiment was conducted in November when the sun is
not as intense as it is in the summer.

\section{Conclusion}
In this experiment, we built a Dye Sensitized Solar Cell (DSSC) which was able
to convert light energy into electrical energy. The testing of the DSSC showed
that it was able to produce a maximum power of 0.040 mW under tungsten light,
which is a very small amount of power. This could be due to the small surface
area of the solar cell, low efficiency of the dye or improper assembly of the
solar cell. However, the experiment was successful in demonstrating the working
of a DSSC which can be a sustainable source of energy in the future.

\section{Questions}
\begin{enumerate}
	\item What is the function of leaves juice?

	      \textbf{Answer:} The leaves juice acts as a dye that absorbs light and
	      converts it into electrical energy.

	\item What is the purpose to coat the ITO substrate with graphite pencil?

	      \textbf{Answer:} The graphite pencil is used to coat the ITO substrate with
	      a layer of carbon. This is done to provide a conductive surface for the
	      electrons to flow through.

	\item Define Voltage, Current and Power?

	      \textbf{Answer:} Voltage is the potential difference between two points in
	      an electric field. Current is the rate of flow of charge. Power is the rate
	      of energy transfer i.e $P = VI$.

	\item Did your solar cell work? What is the maximum power produced?

	      \textbf{Answer:} Yes, our solar cell worked. But the maximum power produced was
	      only 0.040 mW.

	\item How would you assemble it to measure voltage and current?

	      \textbf{Answer:} We would connect the positive terminal of the multimeter
	      to the ITO side of the solar cell and the negative terminal to the carbon
	      side. We would then measure the voltage and current.

	\item How could you improve the efficiency of your solar cell?

	      \textbf{Answer:} We could improve the efficiency of our solar cell by
	      increasing the surface area of the solar cell. This would allow more light
	      to be absorbed and converted into electrical energy.
\end{enumerate}

\end{document}