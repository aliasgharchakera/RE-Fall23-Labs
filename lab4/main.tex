\documentclass[a4paper, 12pt, english]{article}

% \usepackage[portuges]{babel}
\usepackage[utf8]{inputenc}
\usepackage{amsmath,amssymb}
\usepackage{graphicx}
\usepackage{subfig}
\usepackage[colorinlistoftodos]{todonotes}

\usepackage{indentfirst}
\usepackage{verbatim}
\usepackage{textcomp}
\usepackage{gensymb}
\usepackage{float}

\usepackage{relsize}

\usepackage{lipsum}% http://ctan.org/pkg/lipsum
\usepackage{xcolor}% http://ctan.org/pkg/xcolor
\usepackage{xparse}% http://ctan.org/pkg/xparse
\NewDocumentCommand{\myrule}{O{1pt} O{2pt} O{black}}{%
	\par\nobreak % don't break a page here
	\kern\the\prevdepth % don't take into account the depth of the preceding line
	\kern#2 % space before the rule
	{\color{#3}\hrule height #1 width\hsize} % the rule
	\kern#2 % space after the rule
	\nointerlineskip % no additional space after the rule
}
\usepackage[section]{placeins}

\usepackage{booktabs}
\usepackage{colortbl}%
\newcommand{\myrowcolour}{\rowcolor[gray]{0.925}}

\usepackage[obeyspaces]{url}
\usepackage{etoolbox}
\usepackage[colorlinks,citecolor=black,urlcolor=blue,bookmarks=false,hypertexnames=true]{hyperref}

\usepackage{geometry}
\geometry{
	paper=a4paper, % Change to letterpaper for US letter
	inner=3cm, % Inner margin
	outer=3cm, % Outer margin
	bindingoffset=.5cm, % Binding offset
	top=2cm, % Top margin
	bottom=2cm, % Bottom margin
	%showframe, % Uncomment to show how the type block is set on the page
}
\usepackage{fancyhdr}

% Define header and footer
\pagestyle{fancy}
\fancyhf{}
\lhead{ENER 104L}
\rhead{iSciM, Habib University} % Right-aligned page number in the header
\rfoot{\thepage} % Right footer text
%*******************************************************************************%
%************************************START**************************************%
%*******************************************************************************%
\begin{document}

%************************************TITLE PAGE**************************************%
\begin{titlepage}
	\begin{center}
		\textbf{\LARGE Habib University}\\[0.5cm]
		\textbf{\large iSciM}\\[0.2cm]
		\textbf {\large Fall 2023}\\[0.2cm]
		\vspace{20pt}
		\includegraphics[width=5cm]{../habiblogo.jpg}\\[1cm]
		\par
		\vspace{20pt}
		\textbf{\Large ENER 104L RENEWABLE ENERGY}\\
		\vspace{15pt}
		\myrule[1pt][7pt]
		\textbf{\LARGE  LABORATORY REPORT 4}\\
		\vspace{15pt}
		\textbf{\large Parabolic Trough}\\
		\myrule[1pt][7pt]
		\vspace{25pt}
		\begin{tabular}{@{}p{5cm}p{3cm}@{}}
			\textbf{\large Student Name} & \textbf{\large Student ID} \\
			Ali Asghar Yousuf            & ay06993                    \\ % No1 
			Syed Ibrahim Ali Haider      & sh06565                    \\ % No2
		\end{tabular}

		\vspace{10pt}
		\begin{tabular}{@{}p{5cm}p{3cm}@{}}
			\textbf{\large Group Name} & \textbf{\large Group No.} \\
			Insane Fr                  & 1                         \\
		\end{tabular}

		\vspace{45pt}
		\textbf {\large Lab Instructors:}\\[0.2cm]
		\Large {Paishwa Naqvi}\\[0.1cm]
		\Large {Amber Talat}\\[0.1cm]
		\Large {Mah Noor Jamil}\\[0.1cm]
	\end{center}

	\par
	\vfill
	\begin{center}
		\textbf{\today}\\
	\end{center}

\end{titlepage}

%************************************TABLE OF CONTENTS**************************************%

%  %Sumário
%  \newpage
%  \tableofcontents
%  \thispagestyle{empty}
%  %End Sumário

%********************************%
%***********SECTION 1************%
%********************************%
\newpage
\section{Objectives}
\begin{itemize}
	\item To understand the working of a parabolic trough.
	\item To conduct an experiment to determine the effect of the angle of incidence on
	      the temperature of the water.
	\item To understand the effect of placement of the test tube wrt the parabolic trough
	      on the temperature of the water.
\end{itemize}

\section{Abstract}
This experiment was conducted to understand the working of a parabolic trough.
The experiment was conducted in two parts. In the first part, the effect of the
angle of incidence on the temperature of the water was determined. In the
second part, the effect of placement of the test tube wrt the parabolic trough
on the temperature of the water was determined.
\section{Result and Analysis}
\subsection{Part A}
\subsubsection{Experiment 1}
\begin{table}[H]
	\caption{\label{tab:Table 1} Temperature Table}
	\centering
	\begin{tabular}{c c}
		\toprule
		\textbf{Time (min)}
		   & \textbf{Temperature (\degree C)} \\
		\cmidrule[0.4pt](r{0.125em}){1-1}%
		\cmidrule[0.4pt](lr{0.125em}){2-2}%
		% \midrule
		0  & 26                               \\
		2  & 27                               \\
		4  & 27                               \\
		6  & 27                               \\
		8  & 27                               \\
		10 & 27                               \\
		12 & 27                               \\
		\bottomrule
	\end{tabular}
\end{table}
We can see that the temperature of the water remains constant at 27\degree C for the
entire duration of the experiment. The 1\degree C increase in temperature can be
attributed as an error in the experiment.

\subsubsection{Experiment 2}
\begin{table}[H]
	\caption{\label{tab:Table 2} Temperature Table}
	\centering
	\begin{tabular}{c c}
		\toprule
		\textbf{Time (min)}
		   & \textbf{Temperature (\degree C)} \\
		\cmidrule[0.4pt](r{0.125em}){1-1}%
		\cmidrule[0.4pt](lr{0.125em}){2-2}%
		% \midrule
		0  & 26                               \\
		2  & 28                               \\
		4  & 29                               \\
		6  & 30                               \\
		8  & 31                               \\
		10 & 32                               \\
		12 & 33                               \\
		\bottomrule
	\end{tabular}
\end{table}
We can see that the temperature of the water increases by 7\degree C in 12 minutes.
This is a significant increase in temperature and can be attributed to the
the test tube being placed at the focal point of the parabolic trough.

\subsection{Part B}
\subsubsection{Experiment 1}
\begin{table}[H]
	\caption{\label{tab:Table 3} Temperature Table}
	\centering
	\begin{tabular}{c c}
		\toprule
		\textbf{Time (min)}
		   & \textbf{Temperature (\degree C)} \\
		\cmidrule[0.4pt](r{0.125em}){1-1}%
		\cmidrule[0.4pt](lr{0.125em}){2-2}%
		% \midrule
		0  & 26                               \\
		2  & 28                               \\
		4  & 29                               \\
		6  & 29                               \\
		8  & 30                               \\
		10 & 31                               \\
		12 & 31                               \\
		\bottomrule
	\end{tabular}
\end{table}
We can see a gradual increase in temperature of the water by 5\degree C in 12
minutes but not as significant as the previous experiment. This can be attributed
to the test tube being placed farther from the focal point of the parabolic trough.

\subsubsection{Experiment 2}
\begin{table}[H]
	\caption{\label{tab:Table 3} Temperature Table}
	\centering
	\begin{tabular}{c c}
		\toprule
		\textbf{Time (min)}
		   & \textbf{Temperature (\degree C)} \\
		\cmidrule[0.4pt](r{0.125em}){1-1}%
		\cmidrule[0.4pt](lr{0.125em}){2-2}%
		% \midrule
		0  & 25                               \\
		2  & 27                               \\
		4  & 29                               \\
		6  & 30                               \\
		8  & 32                               \\
		10 & 33                               \\
		12 & 34                               \\
		\bottomrule
	\end{tabular}
\end{table}
We see a significant increase in temperature of the water by 9\degree C in 12
minutes. This can be attributed to the test tube being placed at the focal point
of the parabolic trough and closer to the light source.
\section{Conclusion}
We can conclude that the temperature of the water increases with the increase in
the angle of incidence of the light source on the parabolic trough. We can also
conclude that the temperature of the water increases with the decrease in the
distance of the test tube from the focal point of the parabolic trough. Lastly, we
can say that the temperature of the water increases with the decrease in the
distance of the test tube from the light source.

\end{document}